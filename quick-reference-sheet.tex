\documentclass[twocolumn, 10pt]{scrartcl}
\title{Quick reference sheet for the \texttt{phimisci} document class}
\date{}
\author{}

% Document layout
% Narrow margins
\usepackage[margin=.75cm]{geometry}
\columnsep=1cm

% Tiny spacing for the titling
\usepackage{titling}
  \preauthor{}
  \postauthor{}
  \predate{}
  \postdate{}
  \renewcommand{\droptitle}{-20pt}
  \renewcommand{\maketitlehooka}{\bfseries}

% Fonts
\usepackage{tgheros}
\usepackage[zerostyle=d]{newtxtt}
\usepackage{pifont}
\let\familydefault\sfdefault

% Computer code
\usepackage{listings}
\lstset{basicstyle=\small\ttfamily,
        keywordstyle=\small\ttfamily,
        language=[LaTeX]TeX,
        frame=single
        }

% Commands to print LaTeX commands and their arguments
\NewDocumentCommand{\Command}{m}{\texttt{\textbackslash #1}}
\NewDocumentCommand{\OArg}{m}{\texttt{[#1]}}
\NewDocumentCommand{\MArg}{m}{\texttt{\{#1\}}}
\NewDocumentCommand{\Meta}{m}
  {\texttt{$\langle$\textit{#1}$\rangle$}}
        
% Better description environments
\usepackage{enumitem}
\usepackage{booktabs}

\begin{document}
\maketitle

\addsec{A minimal example document}
\begin{lstlisting}
\documentclass{phimisci}
\PhiMiSciSettings
  {
    volume = 10,
    doi = 10.33735/phimisci.2100.1234
  }
\author{A. U. Thor
        \affiliation{Great Institution}
        \and E. di Tor\affiliation{}}
\title[Short title for running head]
      {My long and eloquent title}
\keywords{philosophy;mind;thinking}
\begin{document}
    \maketitle
    ...
    \printbibliography
\end{document}
\end{lstlisting}

\addsec{How to input meta data}
\begin{description}[style=nextline]
    \item[\Command{author}
          \OArg{\Meta{Short name}}
          \MArg{\Meta{Author data}}
         ]
    In \MArg{\Meta{Author data}}, separate individuals by \Command{and}:
\begin{lstlisting}
\author{A. U. Thor \and E. di Tor}
\end{lstlisting}

Each author can contain an \Command{affiliation}\MArg{\Meta{List of affiliations}} and\slash or an \Command{orcid}\MArg{\Meta{id}} command.

Affiliations are separated by semicolons:
\begin{lstlisting}
\author
  {
    Author 1\affiliation{Uni 1; Uni 2}
    \and Author 2\affiliation{Uni 2; Uni 3}
    \and Author 3
  }
\end{lstlisting}
Affiliations are recognised as the same institution only if their names match \emph{exactly}. ``MIT'' and ``M.I.T.'' would not be considered identical.

The \Command{orcid}\MArg{\Meta{id}} is provided verbatim:

\begin{lstlisting}
\author
  {
    Author 1\affiliation{Uni 1; Uni 2}
      \orcid{0000-1111-2222-3333}
    \and Author 2\affiliation{Uni 2; Uni 3}
  }
\end{lstlisting}
The optional \OArg{\Meta{Short name}} argument overwrites the author name(s) printed in the running head.

\item[\Command{title}\OArg{\Meta{short}}\MArg{\Meta{full}}] Uses the \Meta{full} title in all places a title is used in the document. The running head title can be overwritten by providing the \OArg{\Meta{short}} argument.

\item[\Command{subtitle}\MArg{\Meta{subtitle}}] The subtitle, for papers that have one.

\item[\Command{keywords}\MArg{\Meta{List of keywords}}] Keywords in the list should be separated by a semicolon. 

\item[\Command{contact}\OArg{\Meta{mode}}\MArg{\Meta{contact name}}\MArg{\Meta{contact details}}] Supply contact details for the primary contact. The \Meta{contact name} should match one of the authors \emph{exactly}. \Meta{contact details} are interpreted as an e-mail address as default. When \OArg{\Meta{mode}} is set to \texttt{website}, they are interpreted as a website instead. When \OArg{\Meta{mode}} is set to any other value, \Meta{contact details} are ouput verbatim.

\item[\Command{PhiMiSciSettings}\MArg{key\textsubscript{1} = value\textsubscript{1}, key\textsubscript{2} = value\textsubscript{2}, ...}]
Accepts further meta data settings and their values. Possible settings are:
  \begin{description}[style=sameline]
      \item[\texttt{year}] The publication year (defaults to the current year at time of compilation)
      \item[\texttt{volume}] The publication volume. Defaults to the value of the \emph{current} year minus 2019. Has to be adjusted manually if \texttt{year} is given.
      \item[\texttt{doi}] DOI in full format, without default (e.g., \texttt{10.33735\slash phimisci.2000.12345})
      \item[\texttt{issue}] Title of a special issues (optional, default empty)
      \item[\texttt{editor}] Editor(s), separated by a semicolon, of a special issue (empty by default). Ignored if \texttt{issue} is empty or not supplied.
      \item[\texttt{discussed-book}] Title of the discussed book in symposia (optional, empty by default). Is ignored if \texttt{issue} is given.
      \item[\texttt{discussed-book-authors}] The authors of the book in a symposium, separated by semicola. Ignored if \texttt{discussed-book} not given.
      \item[\texttt{language}] A comma-separated list of language(s) used in the paper. The last item is the main language. Defaults to \texttt{english}.
      \item[\texttt{stage}] The current production step. Can be either of \texttt{preparation} (the default), \texttt{submission}, \texttt{draft} or \texttt{final}.
  \end{description}
\end{description}
\addsec{Elements configured in the preamble}
\textit{Note:} The following commands must be placed in the preamble, before \Command{begin}\MArg{document}. 
\begin{description}[style=nextline]
    \item[\Command{abstract}\MArg{\Meta{text}}] Print \Meta{text} as the paper's abstract.
    \item[\Command{acknowledgments}\MArg{\Meta{thanks}}] Print \Meta{thanks} in a separate section, at the end of the paper (optional; not printed if not supplied or left empty).
    \item[\Command{dedication}\MArg{\Meta{text}}] Print \Meta{text} as the dedication on the title page (different from \Command{dictum}).
\end{description}

\pagebreak
\addsec{Elements configured in the document}
\begin{description}[style=nextline]
    \item[\Command{PhiMiSciParagraphNumber}\MArg{\Meta{n}}] Print \Meta{n} as the paragraph number in the outer margin.
    \item[\Command{dictum}\OArg{\Meta{source}}\MArg{\Meta{epigraph}}] An epigraph. The \Meta{source} is printed below the \Meta{epigraph}. Can be used multiple times in one paper.
\end{description}

\addsec{Production stages \& their output}

 \begin{tabular}{lcccc}
     \toprule
   &            & \multicolumn{3}{c}{Output ... ?}\\\cmidrule(lr){3-5}
     Stage & Overf. lin. & Authors & Contact & Footer\\\midrule
     \ttfamily preparation & \ding{55} & \ding{51} & \ding{51} & \ding{55}\\
     \ttfamily submission & \ding{55} & \ding{55} & \ding{55} & \ding{55}\\
     \ttfamily draft & \ding{51} & \ding{51} & \ding{51} & \ding{51}\\
     \ttfamily final & \ding{55} & \ding{51} & \ding{51} & \ding{51}\\
     \bottomrule
     \end{tabular}

\addsec{General settings}
These settings are accessible in the \texttt{settings} folder via {\Command{PhiMiSciSettings}\MArg{settings/\Meta{key\textsubscript{1}} = \Meta{value\textsubscript{1}}, ...}}. Values following equation signs are the initial values.

\begin{description}[style=nextline,font=\ttfamily]
    \item[author-output-separator = \{, \}] Separator between two authors on the title page.  
    \item[author-output-final-separator = \{\textbackslash\& \}] Separator between final two authors on the title page.
    \item[affiliations-input-separator = \{;\}] Separator between affiliations within \Command{affiliation}.
    \item[citation-file = phimisci-current-article.bib] 
    Name for the automatically generated \texttt{.bib} file that contains the data of the present article.
    \item[copyright-text = \Meta{license text}]
    Copyright information text
    \item[emergency-stretch = 2em] Maximally allowed inter word spacing, to be used only when no ideal paragraph composition can be found.
    \item[extra-sentence-spacing = false] Whether or not to print extra spacing after sentence punctuation (can be either \texttt{true} or \texttt{false}).
    \item[dictum-width = 0.62\Command{textwidth}] The width of dictums entered by \Command{dictum} (a length is expected).
    \item[footnote-break-penalty = 1000] The penalty for breaking footnotes across pages (an integer is expected, max. value \texttt{10000}.)
    \item[logo-url = \{\}] File path to the journal's logo. The path can be relative to the current working directory.
    \item[logo-width = 20.6mm] The width at which the journal's logo is printed. The height is determined automatically from this width.
    \item[submission-footer-text = \Meta{Submission text}] A text printed in the footer if \texttt{stage} is set to \texttt{submission}.
    \item[keyword-input-separator = ;] The separator between items in \Command{keywords}
    \item[keyword-output-separator = \{%
      \Command{nobreak}\Command{space}
      \Command{textsf\MArg{\Command{textbullet}}}
      \Command{space}%
    \}] The output separator between keywords
    \item[number-authors-header = 2] Maximum numbers of authors to be printed in the running head before substitution by \emph{et al.} is activated (see also \texttt{locale/et-al})
    \item[orcid-color = A6CE39] The colour of the ORCiD icon next to authors. An HTML colour code is expected.
    \item[orphan-penalty = 300] A penalty for orphans. An integer (max \texttt{10000}) is expected, but note that \texttt{phimisci} uses \texttt{sloppy-bottom} to avoid orphans. Only change this if you know how to fight dragons, and please set \verb+sloppy-bottom=false+ before you continue.
    \item[output-orcids = true] Whether to print ORCiDs (a Boolean is expected)
    \item[output-bibliography = true] Whether or not to print the bibliography automatically at the end of the document. Disable for manual use of \Command{printbibliography}. 
    \item[sloppy-bottom = true] An advanced mechanic to avoid widows and orphans (originally due to Donald Arseneau).
    \item[widow-penalty = 500] A penalty against widows. Please read the note regarding \texttt{orphan-penalty}.
\end{description}

\addsec{Localisation}

The following names for objects can be configured via  \Command{PhiMiSciSettings}\MArg{locale/\Meta{key\textsubscript{1}} = \Meta{value\textsubscript{1}}, ...}.

\begin{description}[font=\ttfamily]
    \item[abstract] The name of the \emph{Abstract} section on the title page
    \item[acknowledgments] The name of the \emph{Acknowledgments} at the back of the paper (Note Acknowledg\textbf{\emph{e}}ments in BrE).
    \item[contact] Descriptor for contact on title page
    \item[et-al] String appended to first author if number of authors exceeds \texttt{settings/number-authors-header}  (used in the header)
    \item[edited-by] String between SI title and editors on title page 
    \item[authored-by] String between BS title and book author on title page
    \item[journal-name] The name of the journal, used in various places
    \item[journal-short-name] Short name of the journal
    \item[keywords] Descriptor for the \emph{Keywords} on the title page
    \item[volume] Abbreviation for volume on title page header.
    \item[special-issue] Descriptor for SI on title page
    \item[book-symposium] Descriptor for BS on title page
\end{description}
\end{document}
