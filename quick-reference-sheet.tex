\documentclass[twocolumn, 10pt]{scrartcl}
\title{Quick reference sheet for the \texttt{phimisci} document class}
\date{}
\author{}

% Document layout
% Narrow margins
\usepackage[margin=1cm]{geometry}
\columnsep=12pt

% Tiny spacing for the titling
\usepackage{titling}
  \preauthor{}
  \postauthor{}
  \predate{}
  \postdate{}
  \renewcommand{\droptitle}{-20pt}
  \renewcommand{\maketitlehooka}{\bfseries}

% Fonts
\usepackage{tgheros}
\usepackage[zerostyle=d]{newtxtt}
\usepackage{pifont}
\let\familydefault\sfdefault

% Computer code
\usepackage{listings}
\lstset{basicstyle=\small\ttfamily,
        language=[LaTeX]TeX,
        frame=single}

% Commands to print LaTeX commands and their arguments
\NewDocumentCommand{\Command}{m}{\texttt{\textbackslash #1}}
\NewDocumentCommand{\OArg}{m}{\texttt{[#1]}}
\NewDocumentCommand{\MArg}{m}{\texttt{\{#1\}}}
\NewDocumentCommand{\Meta}{m}
  {\texttt{$\langle$\textit{#1}$\rangle$}}
        
% Better description environments
\usepackage{enumitem}

\usepackage{booktabs}

\begin{document}
\maketitle

\addsec{A minimal example document}
\begin{lstlisting}
\documentclass{phimisci}
\PhiMiSciSettings
  {
    volume = 10,
    doi = 10.33735/phimisci.2100.1234
  }
\author{A. U. Thor
        \affiliation{Great Institution}
        \and E. di Tor\affiliation{}}
\title[Short title for running head]
      {My long and eloquent title}
\keywords{philosophy;mind;thinking}
\begin{document}
    \maketitle
    ...
    \printbibliography[heading=phimisci]
\end{document}
\end{lstlisting}

\addsec{How to input meta data}
\begin{description}[style=nextline]
    \item[\Command{author}
          \OArg{\Meta{Short name}}
          \MArg{\Meta{Author data}}
         ]
    In \MArg{\Meta{Author data}}, separate individuals by \Command{and}:
\begin{lstlisting}
\author{A. U. Thor \and E. di Tor}
\end{lstlisting}

Each author can contain an \Command{affiliation}\MArg{\Meta{List of affiliations}} and\slash or an \Command{orcid}\MArg{\Meta{id}} command.

Affiliations are separated by semicolons:
\begin{lstlisting}
\author
  {
    Author 1\affiliation{Uni 1; Uni 2}
    \and Author 2\affiliation{Uni 2; Uni 3}
    \and Author 3
  }
\end{lstlisting}
Affiliations are recognised as the same institution only if their names match \emph{exactly}. ``Ruhr-Universität Bochum'' and ``Ruhr University Bochum'' are two separate institutions as far as \texttt{phimisci} is concerned.

The \Command{orcid}\MArg{\Meta{id}} is provided verbatim:

\begin{lstlisting}
\author
  {
    Author 1\affiliation{Uni 1; Uni 2}
      \orcid{0000-1111-2222-3333}
    \and Author 2\affiliation{Uni 2; Uni 3}
  }
\end{lstlisting}
The optional argument for \Command{author}, \OArg{\Meta{Short name}} overwrites the author name(s) printed in the running head.
\item[\Command{title}\OArg{\Meta{short}}\MArg{\Meta{full}}] Uses the \Meta{full} title in all places a title is used in the document. Optionally, the running head title can be overwritten by providing the \OArg{\Meta{short}} argument.
\item[\Command{keywords}\MArg{\Meta{List of keywords}}] Keywords in the list should be separated by a semicolon. 
\item[\Command{PhiMiSettings}\MArg{key\textsubscript{1} = value\textsubscript{1}, key\textsubscript{2} = value\textsubscript{2}, ...}]
Accepts further meta data settings and their values. Possible settings are:
  \begin{description}[style=sameline]
      \item[\texttt{year}] The publication year (defaults to the current year at time of compilation)
      \item[\texttt{volume}] The publication volume (no default)
      \item[\texttt{doi}] DOI in the format \texttt{10.33735\slash phimisci.2000.12345} (no default)
      \item[\texttt{issue}] Specifies the title of the special issues for papers that appear in one (empty by default)
      \item[\texttt{editor}] Editor(s), separated by a semicolon, of the special issue. Ignored if \texttt{issue} is empty or not supplied (empty by default).
      \item[\texttt{language}] A comma-separated list of language(s) used in the paper. The last item is the main language. Defaults to \texttt{english}.
      \item[\texttt{stage}] The current production step. Can be either of \texttt{preparation} (the default), \texttt{submission}, \texttt{draft} or \texttt{final}.
      \item[\texttt{type}] The publication type. Can be either of \texttt{paper} (default), \texttt{book-review}, \texttt{book-symposium}, \texttt{correction}, \texttt{introduction}. Can be empty or a custom string.
  \end{description}
\end{description}
\addsec{Elements configured in the preamble}
\textit{Note:} The following commands must be placed in the preamble, before \Command{begin}\MArg{document}. These commands will respect paragraph breaks if their content contains them.
\begin{description}[style=nextline]
    \item[\Command{abstract}\MArg{\Meta{text}}] Print \Meta{text} as the paper's abstract.
    \item[\Command{acknowledgments}\MArg{\Meta{thanks}}] Print \Meta{thanks} in a separate section, at the end of the paper (optional; not printed if not supplied or left empty).
\end{description}


\addsec{Elements configured in the document}
\begin{description}[style=nextline]
    \item[\Command{PhiMiSciParagraphNumber}\MArg{\Meta{n}}] Print \Meta{n} as the paragraph number in the right margin.
    \item[\Command{dictum}\OArg{\Meta{source}}\MArg{\Meta{epigraph}}] An epigraph. The \Meta{source} is printed below the \Meta{epigraph}. Can be used multiple times in one paper.
\end{description}

\addsec{Production stages \& their output}

 \begin{tabular}{lcccc}
     \toprule
   &            & \multicolumn{3}{c}{Output ... ?}\\\cmidrule(lr){3-5}
     Stage & Overf. lin. & Authors & Contact & Footer\\\midrule
     \ttfamily preparation & \ding{55} & \ding{51} & \ding{51} & \ding{55}\\
     \ttfamily submission & \ding{55} & \ding{55} & \ding{55} & \ding{55}\\
     \ttfamily draft & \ding{51} & \ding{51} & \ding{51} & \ding{51}\\
     \ttfamily final & \ding{55} & \ding{51} & \ding{51} & \ding{51}\\
     \bottomrule
     \end{tabular}

\addsec{Complete list of settings}
These settings are accessible in the \texttt{settings} folder via {\Command{PhiMiSciSettings}\MArg{settings/\Meta{key\textsubscript{1}} = \Meta{value\textsubscript{1}}, ...}}. Values following equation signs are the initial values.

\begin{description}[style=nextline,font=\ttfamily]
    \item[author-output-separator = \{, \}] Separator between two authors on the title page.  
    \item[author-output-final-separator = \{\textbackslash\& \}] Separator between final two authors on the title page.
    \item[affiliations-input-separator = \{;\}] Separator between affiliations within \Command{affiliation}.
    \item[affiliations-output-separator = \{ \}]
    Output separator between affiliations
    \item[affiliations-output-separator-single-author = \{, \}]
    Output separator between affiliations if there is only one author (no indexes printed)
    \item[affiliations-output-separator-single-author\-last-two = \{\textbackslash\& \}]
    Output separator between final two affiliations if there is only one author (no indexes printed)
    \item[affiliations-output-id-name-separator = \{\}]
    Separator between superscripted index and affiliation name in the output
    \item[affiliations-output-multiple-separator = \{,\}] Separator between multiple affiliation indexes (next to author name)
    \item[copyright-text = \Meta{CC Tet}]
    Copyright information text
    \item[emergency-stretch = 0.5em] Maximally allowed inter word spacing, to be used only when no ideal paragraph composition can be found by \TeX (A length is expected).
    \item[extra-sentence-spacing = false] Whether or not to print extra spacing after sentence punctuation (can be either \texttt{true} or \texttt{false}).
    \item[dictum-width = 0.62\Command{textwidth}] The width of dictums entered by \Command{dictum} (a length is expected).
    \item[footnote-break-penalty = 1000] The penalty for breaking footnotes across pages (an integer is expected, max. value \texttt{10000}.)
    \item[logo-url = \{\}] File path to the journal's logo. The path can be relative to the current working directory.
    \item[logo-width = 2.5cm] The width at which the journal's logo is printed. The height is determined automatically from this width.
    \item[submission-footer-text = \Meta{Submission text}] A text printed in the footer if \texttt{stage} is set to \texttt{submission}.
    \item[keyword-input-separator = ;] The separator between items in \Command{keywords}
    \item[keyword-output-separator = \{ \Command{textperiodcentered\{\}}\}] The output separator between keywords
    \item[number-authors-header = 2] Maximum numbers of authors to be printed in the running head before substitution by \emph{et al.} is activated
    \item[orcid-color = 000000] The colour of the ORCiD icon next to authors. An HTML colour code is expected.
    \item[orphan-penalty = 300] A penalty for orphans. An integer (max \texttt{10000}) is expected, but note that \texttt{phimisci} uses \texttt{sloppy-bottom} to avoid orphans. Only change this if you know how to fight dragons, and please set \verb+sloppy-bottom=false+ before you continue.
    \item[output-orcids = true] Whether to print ORCiDs (a Boolean is expected)
    \item[output-bibliography = true] Whether or not to print the bibliography automatically at the end of the document (a Boolean is expected). 
    \item[sloppy-bottom = true] An advanced mechanic to avoid widows and orphans (originally due to Donald Arseneau).
    \item[widow-penalty = 500] A penalty against widows. Please read the note regarding \texttt{orphan-penalty}.
\end{description}

\addsec{List of fonts}

These fonts can be set using \Command{setkomafont}\MArg{\Meta{font}}.

\begin{description}[font=\ttfamily]
    \item[title] Title on the first page
    \item[author] Author(s) on the title page.
    \item[subject] Type of publication (paper, correction, etc.)
    \item[PhiMiSciAffiliationLine] Affiliations below the author(s)
    \item[PhiMiSciMetadataItem] The meta data elements in the list on the title page (e.g., abstract, rights \& permissions, etc.)
    \item[pageheadfoot] Header and footer contents on all but the first page.
    \item[dictum] Epigraphs (whereever the command \Command{dictum} is used).
    \item[dictumauthor] The author below epigraphs (as input in \Command{dictum}\OArg{\Meta{author}}\Meta{epigraph})
    \item[descriptionlabel] The label in \texttt{description} environments, i.e., the font for the \Meta{label} in \Command{item}\MArg{\Meta{label}}.
    \item[notecolumn.marginpar] The paragraph numbers in the margin.
    \item[section, subsection, subsubsection] The respective section heading
    
    
\end{description}



\end{document}